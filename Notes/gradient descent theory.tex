\documentclass{article}
\usepackage{amsmath, amssymb, amsthm}
\usepackage{algorithm, algpseudocode}

\title{Gradient Descent: Theory}
\author{Mikhailova Olena}
\date{27.03.2025}

\newtheorem{theorem}{Theorem}
\newtheorem{lemma}[theorem]{Lemma}
\newtheorem{definition}[theorem]{Definition}

\begin{document}

\maketitle

\section{Gradient Descent}
\begin{definition}[Gradient Descent]
An iterative optimization algorithm for minimizing a differentiable function \( f: \mathbb{R}^n \to \mathbb{R} \). Starting from an initial point \( x_0 \), it updates:
\[
x_{k+1} = x_k - \eta_k \nabla f(x_k)
\]
where \( \eta_k > 0 \) is the step size (learning rate).
\end{definition}

\textbf{Motivation:} Follows the direction of steepest descent (\(-\nabla f(x)\)) to reduce \( f(x) \). Can be derived from minimizing a first-order Taylor approximation with a quadratic regularization term:
\[
x_{k+1} = \arg\min_y \left\{ f(x_k) + \nabla f(x_k)^T (y - x_k) + \frac{1}{2\eta} \|y - x_k\|^2 \right\}
\]

\section{Descent Lemma}

\subsection{Preliminary Concepts}
\begin{definition}[L-Smoothness]
A function \( f: \mathbb{R}^n \to \mathbb{R} \) is called \textbf{L-smooth} if its gradient is Lipschitz continuous with constant \( L \):
\[
\|\nabla f(x) - \nabla f(y)\| \leq L \|x - y\| \quad \forall x, y \in \mathbb{R}^n
\]
\end{definition}

\subsection{Main Result}
\begin{lemma}[Descent Lemma]
For any L-smooth function \( f \) and \( \forall x, y \in \mathbb{R}^n \):
\[
f(y) \leq f(x) + \nabla f(x)^\top (y - x) + \frac{L}{2} \|y - x\|^2
\]
\end{lemma}

\begin{proof}
\textit{Proof via Taylor expansion and L-smoothness:}

1. Start with Taylor's theorem with integral remainder:
\[
f(y) = f(x) + \int_0^1 \nabla f(x + t(y - x))^\top (y - x) \, dt
\]

2. Subtract linear approximation:
\[
f(y) - f(x) - \nabla f(x)^\top (y - x) = \int_0^1 [\nabla f(x + t(y - x)) - \nabla f(x)]^\top (y - x) \, dt
\]

3. Apply Cauchy-Schwarz and L-smoothness:
\[
\left| f(y) - f(x) - \nabla f(x)^\top (y - x) \right| \leq \int_0^1 L t \|y - x\|^2 \, dt = \frac{L}{2} \|y - x\|^2
\]

4. Combine inequalities to get final result.
\end{proof}

\subsection{Geometric Interpretation}
The Descent Lemma establishes that L-smooth functions:
\begin{itemize}
\item Have \textbf{quadratic upper bounds} on their growth
\item Cannot deviate too far from their linear approximations
\item Permit controlled descent steps in gradient methods
\end{itemize}

\subsection{Connection to Gradient Descent}
For gradient descent update \( y = x - \eta \nabla f(x) \):
\[
f(x_{k+1}) \leq f(x_k) - \eta \|\nabla f(x_k)\|^2 + \frac{L\eta^2}{2} \|\nabla f(x_k)\|^2
\]
\begin{theorem}[Descent Guarantee]
For step size \( \eta \leq 1/L \):
\[
f(x_{k+1}) \leq f(x_k) - \frac{\eta}{2} \|\nabla f(x_k)\|^2
\]
\end{theorem}

\subsection{Special Case: Quadratic Functions}
For \( f(x) = \frac{1}{2}x^\top Q x \) with \( Q \succ 0 \):
\[
f(y) = f(x) + \nabla f(x)^\top (y - x) + \frac{1}{2}(y - x)^\top Q (y - x)
\]
Here \( L = \lambda_{\text{max}}(Q) \), and the Descent Lemma becomes exact.

\subsection{Limitations and Caveats}
\begin{itemize}
\item \textbf{Critical Step Size:} Fails for \( \eta > 1/L \)
\item \textbf{Non-Smooth Functions:} Does not apply to non-differentiable functions
\item \textbf{Local Property:} Only describes local behavior
\end{itemize}

\section{Stepsize Selection}
\subsection{Fixed Stepsize}
\begin{definition}[L-Smoothness]
A function \( f \) is \( L \)-smooth if:
\[
\|\nabla f(x) - \nabla f(y)\| \leq L\|x - y\| \quad \forall x,y
\]
\end{definition}

\begin{theorem}[Optimal Fixed Stepsize]
For \( L \)-smooth functions, GD with \( \eta = \frac{1}{L} \) guarantees:
\[
f(x_{k+1}) \leq f(x_k) - \frac{1}{2L}\|\nabla f(x_k)\|^2
\]
\end{theorem}

\begin{proof}
From the Descent Lemma:
\[
f(x_{k+1}) \leq f(x_k) - \eta\|\nabla f(x_k)\|^2 + \frac{L\eta^2}{2}\|\nabla f(x_k)\|^2
\]
Substitute \( \eta = \frac{1}{L} \):
\[
f(x_{k+1}) \leq f(x_k) - \frac{1}{2L}\|\nabla f(x_k)\|^2 \qedhere
\]
\end{proof}

\begin{example}[Quadratic Function]
For \( f(x) = \frac{1}{2}x^TQx \), optimal stepsize is \( \eta = 2/(\lambda_{\max}(Q) + \lambda_{\min}(Q)) \). With \( \eta = 1/\lambda_{\max}(Q) \), convergence rate becomes:
\[
\|x^k - x^*\| \leq \left(\frac{\kappa-1}{\kappa+1}\right)^k \|x^0 - x^*\|
\]
where \( \kappa = \lambda_{\max}(Q)/\lambda_{\min}(Q) \).
\end{example}

\subsection{Adaptive Stepsize Methods}

\subsubsection{Backtracking Line Search}
\begin{algorithm}[H]
\caption{Backtracking Line Search}
\begin{algorithmic}[1]
\State Initialize \( \eta = \eta_{\text{init}}, \alpha \in (0,0.5), \beta \in (0,1) \)
\While{ \( f(x - \eta\nabla f(x)) > f(x) - \alpha\eta\|\nabla f(x)\|^2 \) }
\State \( \eta \gets \beta\eta \)
\EndWhile
\end{algorithmic}
\end{algorithm}

\textbf{Geometric Interpretation:} Maintains Armijo condition ensuring sufficient decrease:
\begin{equation}
f(x^+) \leq f(x) - \alpha\eta\|\nabla f(x)\|^2
\end{equation}

\begin{theorem}[Convergence with Backtracking]
For \( L \)-smooth \( f \), backtracking GD with \( \eta_{\text{init}} > 0 \) achieves:
\[
\min_{1\leq t\leq k} \|\nabla f(x_t)\|^2 \leq \frac{C(f(x_0) - f_*)}{k}
\]
where \( C \) depends on \( \alpha, \beta \).
\end{theorem}

\subsubsection{Exact Line Search}
\[
\eta_k = \arg\min_{\eta > 0} f(x_k - \eta\nabla f(x_k))
\]

\begin{example}[Rayleigh Quotient]
For quadratic \( f(x) = \frac{1}{2}x^TQx \), exact step:
\[
\eta = \frac{\nabla f(x)^T\nabla f(x)}{\nabla f(x)^T Q \nabla f(x)}
\]
\end{example}

\subsection{Stepsize for Different Function Classes}

\begin{table}[h]
\centering
\begin{tabular}{|l|l|l|}
\hline
\textbf{Function Class} & \textbf{Recommended Stepsize} & \textbf{Convergence Rate} \\
\hline
Non-convex L-smooth & \( \eta = 1/L \) & \( O(1/\sqrt{k}) \) \\
Convex L-smooth & \( \eta = 1/L \) & \( O(1/k) \) \\
\(\mu\)-strongly convex & \( \eta = 2/(\mu + L) \) & Linear \( O(\gamma^k) \) \\
\hline
\end{tabular}
\caption{Stepsize selection guide}
\end{table}



\subsection{Practical Considerations}
\begin{itemize}
\item For unknown \( L \), use backtracking with \( \eta_{\text{init}} = 1 \)
\item Monitor function values: \( f(x_{k+1}) < f(x_k) \)
\item In deep learning: Use adaptive methods (Adam, RMSProp) with step decay
\end{itemize}

\begin{theorem}[Safe Initialization]
For any \( L \)-smooth function, backtracking line search with \( \eta_{\text{init}} \geq 1/L \) will accept \( \eta \geq \beta/L \) within \( \lceil \log_\beta(1/L\eta_{\text{init}}) \rceil \) steps.
\end{theorem}


\section{Metric Projection}
\begin{definition}[Metric Projection]
Let \( C \subseteq \mathbb{R}^n \) be a closed convex set. The metric projection of a point \( x \in \mathbb{R}^n \) onto \( C \) is defined as:
\[
\Pi_C(x) := \arg\min_{y \in C} \|y - x\|_2
\]
This is the unique point in \( C \) closest to \( x \) under the Euclidean norm.
\end{definition}

\subsection{Existence and Uniqueness}
\begin{theorem}[Existence and Uniqueness]
For any closed convex set \( C \subseteq \mathbb{R}^n \) and \( x \in \mathbb{R}^n \), the metric projection \( \Pi_C(x) \) exists and is unique.
\end{theorem}

\begin{proof}
\textbf{Existence:} The function \( f(y) = \|y - x\|^2 \) is coercive and strictly convex. Since \( C \) is closed, a minimizer exists. \\
\textbf{Uniqueness:} Strict convexity of \( f \) guarantees uniqueness. If \( y_1, y_2 \) were both minimizers, then \( \frac{y_1 + y_2}{2} \in C \) (by convexity) would yield a lower function value, contradicting minimality.
\end{proof}

\subsection{Characterizing Inequality}
\begin{theorem}[Projection Inequality]
For any \( x \in \mathbb{R}^n \) and \( y \in C \):
\[
\langle x - \Pi_C(x), y - \Pi_C(x) \rangle \leq 0
\]
Equivalently:
\[
\|x - \Pi_C(x)\|^2 \leq \|x - y\|^2 - \|\Pi_C(x) - y\|^2
\]
\end{theorem}

\begin{proof}
Let \( z = \Pi_C(x) \). From the first-order optimality condition:
\[
\nabla f(z)^T (y - z) \geq 0 \quad \forall y \in C
\]
Since \( \nabla f(z) = 2(z - x) \), this becomes:
\[
\langle z - x, y - z \rangle \geq 0 \quad \Rightarrow \quad \langle x - z, y - z \rangle \leq 0
\]
The equivalence follows by expanding \( \|x - y\|^2 = \|x - z + z - y\|^2 \).
\end{proof}

\subsection{Normal Cone Interpretation}
\begin{theorem}[Normal Cone Characterization]
\( z = \Pi_C(x) \) if and only if:
\[
x - z \in N_C(z)
\]
where \( N_C(z) \) is the normal cone to \( C \) at \( z \):
\[
N_C(z) := \{ v \in \mathbb{R}^n \mid \langle v, y - z \rangle \leq 0 \ \forall y \in C \}
\]
\end{theorem}

\subsection{Examples of Metric Projections}
\begin{property}[Common Projections]
\begin{itemize}
\item \textbf{Affine set}: For \( C = \{x \mid Ax = b\} \):
\[
\Pi_C(x) = x - A^\dagger(Ax - b)
\]
where \( A^\dagger \) is the Moore-Penrose pseudoinverse.

\item \textbf{Non-negative orthant}: For \( C = \mathbb{R}^n_+ \):
\[
(\Pi_C(x))_i = \max(x_i, 0)
\]

\item \textbf{Euclidean ball}: For \( C = \{y \mid \|y\| \leq r\} \):
\[
\Pi_C(x) = \begin{cases}
x & \text{if } \|x\| \leq r \\
r \frac{x}{\|x\|} & \text{otherwise}
\end{cases}
\]
\end{itemize}
\end{property}

\subsection{Projected Gradient Descent}
\begin{algorithm}[H]
\caption{Projected Gradient Descent}
\begin{algorithmic}[1]
\State Initialize \( x_0 \in C \), step size \( \eta > 0 \)
\For{\( k = 0, 1, 2, \ldots \)}
\State \( y_{k+1} = x_k - \eta \nabla f(x_k) \)
\State \( x_{k+1} = \Pi_C(y_{k+1}) \)
\EndFor
\end{algorithmic}
\end{algorithm}

\begin{theorem}[Convergence Guarantee]
If \( f \) is \( L \)-smooth and convex, with \( \eta = 1/L \):
\[
f(x_k) - f_\star \leq \frac{L\|x_0 - x_\star\|^2}{2k}
\]
\end{theorem}

\subsection{Key Lemma for Analysis}
\begin{lemma}[Projection Contraction]
For any \( x \in \mathbb{R}^n \) and \( z \in C \):
\[
\|\Pi_C(x) - z\|^2 \leq \|x - z\|^2 - \|\Pi_C(x) - x\|^2
\]
\end{lemma}

\begin{proof}
Using the characterizing inequality with \( y = z \):
\[
\|x - \Pi_C(x)\|^2 \leq \|x - z\|^2 - \|\Pi_C(x) - z\|^2
\]
Rearranging gives the result.
\end{proof}

\section{Convergence Rates}
\subsection{Non-Convex Functions}
For \( L \)-smooth \( f \), GD achieves:
\[
\min_{0 \leq t \leq k} \|\nabla f(x_t)\|^2 \leq \frac{2L(f(x_0) - f_\star)}{k}
\]
Rate: \( O(1/\sqrt{k}) \).

\subsection{Convex Functions}
For \( L \)-smooth convex \( f \), GD achieves:
\[
f(x_k) - f_\star \leq \frac{L\|x_0 - x_\star\|^2}{2k}
\]
Rate: \( O(1/k) \).

\subsection{Strongly Convex Functions}
For \( \mu \)-strongly convex and \( L \)-smooth \( f \), GD achieves linear convergence:
\[
f(x_k) - f_\star \leq \left(1 - \frac{\mu}{L}\right)^k (f(x_0) - f_\star)
\]
Rate: \( O(\log(1/\epsilon)) \) iterations to reach \( \epsilon \)-accuracy.

\begin{theorem}[Convergence under PL Condition]
If \( f \) satisfies Polyak-Łojasiewicz inequality:
\[
\frac{1}{2}\|\nabla f(x)\|^2 \geq \mu(f(x) - f_\star)
\]
then GD converges linearly even without strong convexity.
\end{theorem}

\end{document}