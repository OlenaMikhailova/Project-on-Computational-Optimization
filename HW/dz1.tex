\documentclass{article}
\usepackage{amsmath, amssymb}
\usepackage{enumitem}
\usepackage{lmodern}
\usepackage{anyfontsize}

\begin{document}

\title{Homework Assignment 1}
\date{Vladyslav Skrynyk}
\maketitle

\begin{itemize}[label=\textbullet, itemsep=10pt]
    \item \textbf{Problem 1}
    
    {\fontsize{12}{14}\selectfont Prove that a strongly convex function has a unique minimum.}
    
    \item \textbf{Solution 1}
    
	Since $f$ is continuous on the compact interval $[a,b]$, by the extreme value theorem, $f$ attains its minimum at some point $x_0 \in [a,b]$. That is,
\[
f(x_0) \leq f(x) \quad \forall x \in [a,b].
\]

Suppose, for the sake of contradiction, that there exists another point $x_1 \in [a,b]$, with $x_0 \neq x_1$, such that $f(x_0) = f(x_1)$, i.e., $x_0$ and $x_1$ are both global minima.

Since $f$ is strictly convex, for any $t \in (0,1)$, the combination $x_t = t x_0 + (1-t) x_1$ satisfies:
\[
f(x_t) < t f(x_0) + (1-t) f(x_1).
\]
Substituting $f(x_0) = f(x_1)$, we get:
\[
f(x_t) < t f(x_0) + (1-t) f(x_0) = f(x_0).
\]
This contradicts the assumption that $f(x_0)$ is a global minimum, as it implies the existence of a point $x_t$ where $f(x_t) < f(x_0)$.

Thus, our assumption that $f$ has more than one global minimum must be false, meaning that $f$ has a unique global minimum. 
  
    
    \item \textbf{Problem 2}
    
    {\fontsize{12}{14}\selectfont Prove that if the gradient of a convex function is zero at a point, then that point is a minimum. Also, show that this does not necessarily hold for a non-convex function.}
    
    \item \textbf{Solution 2}

Let $f: \mathbb{R}^n \to \mathbb{R}$ be a differentiable convex function. We want to show that if $\nabla f(x^*) = 0$, then $x^*$ is a global minimum.

\subsection*{Step 1: First-Order Condition for Convexity}
By the definition of convexity, for any $x, y \in \mathbb{R}^n$,
\[
    f(y) \geq f(x) + \nabla f(x)^T (y - x).
\]

Setting $x = x^*$ and using $\nabla f(x^*) = 0$, we get:
\[
    f(y) \geq f(x^*) \quad \forall y \in \mathbb{R}^n.
\]
Thus, $x^*$ is a global minimum.

\section*{Counterexample for a Non-Convex Function}
Consider the function:
\[
    f(x) = x^3.
\]
Its derivative is:
\[
    \nabla f(x) = 3x^2.
\]
Setting $\nabla f(0) = 0$, we find that $x = 0$ is a critical point. However, $f(x)$ is \textbf{not} convex, and $x = 0$ is neither a minimum nor a maximum, but an inflection point.

Thus, the statement does not hold for non-convex functions.

\end{itemize}

\end{document}